\documentclass[10pt]{beamer}
\usetheme[
%%% options passed to the outer theme
%    progressstyle=movCircCnt,   %either fixedCircCnt, movCircCnt, or corner
%    rotationcw,          % change the rotation direction from counter-clockwise to clockwise
%    shownavsym          % show the navigation symbols
  ]{AAUsimple}
  
% If you want to change the colors of the various elements in the theme, edit and uncomment the following lines
% Change the bar and sidebar colors:
%\setbeamercolor{AAUsimple}{fg=red!20,bg=red}
%\setbeamercolor{sidebar}{bg=red!20}
% Change the color of the structural elements:
% \setbeamercolor{structure}{fg=red}
% Change the frame title text color:
%\setbeamercolor{frametitle}{fg=blue}
% Change the normal text color background:
%\setbeamercolor{normal text}{fg=black,bg=gray!10}
% ... and you can of course change a lot more - see the beamer user manual.

\usepackage[utf8]{inputenc}
\usepackage[danish]{babel}
\usepackage[T1]{fontenc}
% Or whatever. Note that the encoding and the font should match. If T1
% does not look nice, try deleting the line with the fontenc.
\usepackage{helvet}

% colored hyperlinks
\newcommand{\chref}[2]{%
  \href{#1}{{\usebeamercolor[bg]{AAUsimple}#2}}%
}

\title{Introduktion Til \LaTeX}

\subtitle{Mapper og Filer}  % could also be a conference name

\date{\today}

\author{
  %
}

% - Give the names in the same order as they appear in the paper.
% - Use the \inst{?} command only if the authors have different
%   affiliation. See the beamer manual for an example

\institute[
%  {\includegraphics[scale=0.2]{aau_segl}}\\ %insert a company, department or university logo
  Institut for Matematiske Fag\\
  Aalborg Universitet\\
  Danmark
] % optional - is placed in the bottom of the sidebar on every slide
{% is placed on the bottom of the title page
  Institut for Matematiske Fag\\
  Aalborg Universitet\\
  Danmark
  
  %there must be an empty line above this line - otherwise some unwanted space is added between the university and the country (I do not know why;( )
}

% specify a logo on the titlepage (you can specify additional logos an include them in 
% institute command below
\pgfdeclareimage[height=1.5cm]{titlepagelogo}{AAUgraphics/aau_logo_new} % placed on the title page
%\pgfdeclareimage[height=1.5cm]{titlepagelogo2}{AAUgraphics/aau_logo_new} % placed on the title page
\titlegraphic{% is placed on the bottom of the title page
  \pgfuseimage{titlepagelogo}
%  \hspace{1cm}\pgfuseimage{titlepagelogo2}
}

\begin{document}
% the titlepage
{\aauwavesbg%
\begin{frame}[plain,noframenumbering] % the plain option removes the header from the title page
  \titlepage
\end{frame}}
%%%%%%%%%%%%%%%%

% TOC
\begin{frame}{Agenda}{}
\tableofcontents
\end{frame}
%%%%%%%%%%%%%%%%


\section{Introduktion}
\begin{frame}{Introduktion}
  Hvad er \LaTeX\ og hvorfor er det nødvendigt at kende til som matematik studerende?
  \begin{itemize}
  \item<1-> \LaTeX Er et typografisk system som blandt andet er særligt velegnet til at skrive matematiske formler.
  \item<2-> Det er IKKE et WYSIWYG program ligesom Microsoft Word etc.
  \item<3-> Tekst og formler skrives i et `kode-sprog` som først bliver `kompileret` hvorefter der dannes en pdf med resultatet.
  \item<4-> Man kan inkludere ekstra `pakker` (\texttt{\textbackslash usepackage \{pakke-navn \}}) i et \LaTeX dokument med flere funktionaliteter (med pakken \texttt{tikz} kan man generere flotte figurer).
  \item<5-> Det kræver tilvænning at skrive i \LaTeX; i starten vil man opleve `kompilerings-fejl` og man skal derfor `debugge` sit dokument.
  \end{itemize}
\end{frame}

\begin{frame}[fragile]{Introduktion}{Hello World!}
  \begin{block}{master.tex}
\begin{verbatim}
  \documentclass[10pt, danish]{article}
  \begin{document}
  Hello World! I present to you Eulers identity:
  \begin{equation}
    \label{eq:eulers_identity}
    e^{i\pi} + 1 = 0.
  \end{equation}
  \end{document}
\end{verbatim}    
  \end{block}
  \begin{block}{master.pdf}
    Hello World! I present to you Eulers identity:
  \begin{equation}
  \label{eq:eulers_identity}
  e^{i\pi} + 1 = 0.
\end{equation} 
  \end{block}
\end{frame}

\section{Mapper og Filer}
\begin{frame}{Mapper og Filer}
  Det er vigtigt, at have en mappestruktur på sin computer.

\end{frame}

% \section{Introduction}
% % motivation for creating this theme
% \begin{frame}{Introduction}{}
% \begin{block}{Why the AAU Simple beamer theme?}
%   \begin{itemize}
%     \item<1-> During the last couple of years, I have shared the beamer themes named \chref{http://kom.aau.dk/~jkn/latex/latex.php\#beamer_aausidebar}{AAU Sidebar} and \chref{http://kom.aau.dk/~jkn/latex/latex.php\#beamer_aalborg}{Aalborg} on my website \chref{http://kom.aau.dk/~jkn}{http://kom.aau.dk/\textasciitilde jkn}.
%     \item<2-> Both of these themes feature a sidebar in which the table of content and progress are shown.
%     \item<3-> Some people (in particular one - Yes, I am looking at you, Mads) have been asking about an AAU beamer theme without a sidebar. The present theme named \alert{AAU Simple} is precisely that. 
%     \item<4-> Like the \chref{http://kom.aau.dk/~jkn/latex/latex.php\#beamer_aausidebar}{AAU Sidebar} theme, the theme is not really useful to people not affiliated with AAU due to the tight integration between the theme and the round AAU logo. However, everyone is of course encouraged to download and modify the theme according to their own needs. 
%   \end{itemize}
% \end{block}
% \end{frame}
%%%%%%%%%%%%%%%%


% \subsection{Modifying the theme}
% % how to modify the theme
% {\setbeamercolor{AAUsimple}{fg=gray!50,bg=orange!50}
%  \setbeamercolor{structure}{fg=red}
%  \setbeamercolor{frametitle}{use=structure,fg=structure.fg}
%  \setbeamercolor{normal text}{bg=gray!20}
% \begin{frame}{User Interface}{Modifying the Theme}
%   \begin{itemize}
%     \item<1-> The default configuration of fonts, colors, and layout complies with the \chref{http://aau.designguiden.dk}{AAU design guidelines} and is the \alert{official} version of the theme.
%     \item<2-> However, you can modify specific elements of the theme through the templates provided by the beamer class. Please refer to the beamer user manual for instructions.
%     \item<3-> For example, on this slide the following commands have been used
%       \begin{itemize}
%         \item Change the header colours:\\
%         {\tt \textbackslash setbeamercolor\{AAUsimple\}\{fg=blue!20,bg=red!50\}}
%         \item Change the color of the structural elements:\\
%         {\tt \textbackslash setbeamercolor\{structure\}\{fg=black\}}\\
%         \item Change the frame title text color:
%         {\tt \textbackslash setbeamercolor\{frametitle\}\{use=structure, fg=structure.fg\}}
%         \item Change the background color of the text
%         {\tt \textbackslash setbeamercolor\{normal text\}\{bg=gray!20\}}
%       \end{itemize}
%   \end{itemize}
% \end{frame}}
%%%%%%%%%%%%%%%%

\end{document}
