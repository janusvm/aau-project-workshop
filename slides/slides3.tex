\documentclass[10pt]{beamer}
\usetheme{AAUsimple}

\usepackage[utf8]{inputenc}
\usepackage[T1]{fontenc}
\usepackage[danish]{babel}
\usepackage{helvet}
\usepackage{listings}
\usepackage{tikz}
\usepackage{fontawesome}

\newcommand{\chref}[2]{%
  \href{#1}{{\usebeamercolor[bg]{AAUsimple}#2}}%
}

\newcommand{\FTdir}{}
\def\FTdir(#1,#2,#3){%
  \FTfile(#1,{{\color{black!40!white}\faFolderOpen}\hspace{0.2em}#3})
  (tmp.west)++(0.8em,-0.4em)node(#2){}
  (tmp.west)++(1.5em,0)
  ++(0,-1.3em) 
}

\newcommand{\FTfile}{}
\def\FTfile(#1,#2){%
  node(tmp){}
  (#1|-tmp)++(0.6em,0)
  node(tmp)[anchor=west,black]{\tt #2}
  (#1)|-(tmp.west)
  ++(0,-1.2em) 
}

\newcommand{\FTroot}{}
\def\FTroot{tmp.west}

\author{}
\institute{
  Institut for Matematiske Fag\\
  Aalborg Universitet\\
  Danmark

}
\pgfdeclareimage[height=1.5cm]{titlepagelogo}{AAUgraphics/aau_logo_new}
\titlegraphic{\pgfuseimage{titlepagelogo}}


\title{Bibliografi og \LaTeX \ Præsentationer med Beamer}
\subtitle{Workshop 3}
\date{6. september 2019}

\begin{document}

% Title page
{\aauwavesbg
  \begin{frame}[plain,noframenumbering]
  \titlepage
\end{frame}}

% Table of contents
\begin{frame}{Agenda}{}
\tableofcontents
\end{frame}

% Main content

\section{BibTeX}

\begin{frame}[containsverbatim]{BibTeX}{\texttt{.bib}-fil}
  \begin{itemize}
  \item BibTeX er et modul til \LaTeX\ til at lave kildehenvisninger
  \item Kilder skrives ind i en eller flere \texttt{.bib}-filer
  \item Let at referere fra alle steder i dokumentet
  \end{itemize}

  \begin{block}{incl/bib/references.bib}
\begin{verbatim}
@Book{calc,
  author    = {C. Henry Edwards and David E. Penney},
  title     = {Calculus: Early Transcendentals},
  publisher = {Pearson},
  year      = {2014},
  edition   = {7th}
}
\end{verbatim}
  \end{block}
\end{frame}

\begin{frame}[containsverbatim]{BibTeX}{Henvisninger}
  \begin{block}{master.tex}
\begin{verbatim}
...
\bibliographystyle{apalike}
\bibliography{incl/bib/references.bib}
\end{document}
\end{verbatim}
  \end{block}

  Vi kan nu referere til calculus-bogen ved at skrive
  \begin{itemize}
  \item \texttt{\textbackslash cite\{calc\}} eller \texttt{\textbackslash cite[sidetal]\{calc\}}
  \item \texttt{\textbackslash citep\{calc\}} eller \texttt{\textbackslash citep[sidetal]\{calc\}}
  \end{itemize}
\end{frame}


\section{Beamer-Præsentationer}

\begin{frame}[containsverbatim]{Beamer-Præsentationer}{Introduktion}

  \begin{block}{presentation.tex}
\begin{verbatim}
\documentclass[10pt]{beamer}

\title{En Catchy Titel}
\subtitle{En ligeså fed undertitel}
\date{\today}
\begin{document}

\frame{\titlepage}

\section{Introduktion}

\begin{frame}{Introduktion}{Definitioner}
Her defineres centrale koncepter.
\end{frame}
\end{verbatim}
  \end{block}
\end{frame}

\section{Opgaver}

\begin{frame}[containsverbatim]{Opgaver}{}
  \begin{itemize}
  \item Lav en beamer-præsentation og undersøg, hvordan man kan ændre temaet for præsentationen.\footnote{
      \url{https://en.wikibooks.org/wiki/LaTeX/Presentations}
    }
  \item Brug Google Scholar til at finde BibTeX kode til at referere til `Discrete Mathematics and Its Applications` af Kenneth H. Rosen.
    Indsæt denne i en \texttt{.bib}-fil og inkluder den i præsentationen med \verb!\bibliography!
  \item Lav et slide med en indholdsfortegnelse
  \item Lav en \texttt{preamble.tex} fil og indsæt relevant indhold.
    Indsæt specielt den pakke, der gør det muligt at producere følgende.\footnote{
      \url{https://en.wikibooks.org/wiki/LaTeX/Mathematics},\\
      \hspace{3ex} se tabellen i afsnit \emph{Formatting mathematics symbols}
    }
    \[
      \mathbb{N} \subset \mathbb{Z} \subset \mathbb{Q} \subset \mathbb{R} \subset \mathbb{C}
    \]
  \item Indsæt en reference til Rosen-bogen med \verb!\cite!
  \end{itemize}
\end{frame}

\end{document}

%%% Local Variables:
%%% mode: latex
%%% TeX-master: t
%%% End:
