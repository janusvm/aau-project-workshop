\documentclass[10pt]{beamer}
\usetheme{AAUsimple}

\usepackage[utf8]{inputenc}
\usepackage[T1]{fontenc}
\usepackage[danish]{babel}
\usepackage{helvet}
\usepackage{listings}
\usepackage{tikz}
\usepackage{fontawesome}

\newcommand{\chref}[2]{%
  \href{#1}{{\usebeamercolor[bg]{AAUsimple}#2}}%
}

\newcommand{\FTdir}{}
\def\FTdir(#1,#2,#3){%
  \FTfile(#1,{{\color{black!40!white}\faFolderOpen}\hspace{0.2em}#3})
  (tmp.west)++(0.8em,-0.4em)node(#2){}
  (tmp.west)++(1.5em,0)
  ++(0,-1.3em) 
}

\newcommand{\FTfile}{}
\def\FTfile(#1,#2){%
  node(tmp){}
  (#1|-tmp)++(0.6em,0)
  node(tmp)[anchor=west,black]{\tt #2}
  (#1)|-(tmp.west)
  ++(0,-1.2em) 
}

\newcommand{\FTroot}{}
\def\FTroot{tmp.west}

\author{}
\institute{
  Institut for Matematiske Fag\\
  Aalborg Universitet\\
  Danmark

}
\pgfdeclareimage[height=1.5cm]{titlepagelogo}{AAUgraphics/aau_logo_new}
\titlegraphic{\pgfuseimage{titlepagelogo}}


\title{Projekt Strukturering}
\subtitle{Workshop 2}
\date{5. september 2019}

\begin{document}

% Title page
{\aauwavesbg
  \begin{frame}[plain,noframenumbering]
  \titlepage
\end{frame}}

% Table of contents
\begin{frame}{Agenda}{}
\tableofcontents
\end{frame}

% Main content

\section{Opdeling af Projektfiler}

\begin{frame}[containsverbatim]{Opdeling Projektfiler}{Mappestruktur}
  \begin{columns}
    \begin{column}{0.5\textwidth}
      \begin{tikzpicture}
        \draw[color=black!60!white]
        \FTdir(\FTroot,root,P1){
          \FTfile(root,master.tex)
          \FTfile(root,master.pdf)
          \FTdir(root,incl,incl){
            \FTdir(incl,main,main) {
              \FTfile(main,graph\_theory.tex)
            }
            \FTdir(incl,misc,misc) {
              \FTfile(misc,preface.tex)
              \FTfile(misc,introduction.tex)
            }
            \FTdir(incl,app,app) {
              \FTfile(app,notation.tex)
            }
          }
          \FTdir(root,fig,fig) {
            \FTdir(fig,img,img){
              \FTfile(img,digraph.pdf)
            }
            \FTdir(fig,tab,tab) {
              \FTfile(tab,summary\_table.tex)
            }
          }
        };
      \end{tikzpicture}
    \end{column}
    \begin{column}{0.5\textwidth}
      \begin{block}{}
        Eksempel på dele af en mappestruktur for et P1-projekt.
      \end{block}
      \begin{block}{}
        Forskellige typer af indhold ligger i forskellige undermapper.
      \end{block}
    \end{column}
  \end{columns}
\end{frame}

\begin{frame}[containsverbatim]{Opdeling af Projektfiler}{Inkluder Filer i Masteren}
  \begin{block}{master.tex}
\begin{verbatim}
  \documentclass[10pt]{report}
  \begin{document}
  \include{incl/misc/preface}
  \include{incl/misc/introduction}
  \include{incl/main/graph_theory}
  \appendix
  \include{incl/app/notation}
  \end{document}
\end{verbatim}
  \end{block}

  \begin{block}{graph\_theory.tex}
\begin{verbatim}
\chapter{Grafteori}
Dette kapitel omhandler...
\input{fig/tab/summary_table.tex}
\end{verbatim}
  \end{block}
\end{frame}

\begin{frame}[containsverbatim]{Opdeling af Projektfiler}{Preamble}
  \begin{block}{preamble.tex}
\begin{verbatim}
\usepackage[utf8]{inputenc}           % DK bogstaver ind
\usepackage[T1]{fontenc}              % DK bogstaver ud
\usepackage[danish]{babel}            % DK formattering
\usepackage{mathtools,amssymb}        % Matematik
\usepackage{graphicx}                 % Billedfil-support
\end{verbatim}
  \end{block}
    \begin{block}{master.tex}
\begin{verbatim}
\documentclass[10pt]{report}
\usetheme{AAUsimple}

\usepackage[utf8]{inputenc}
\usepackage[T1]{fontenc}
\usepackage[danish]{babel}
\usepackage{helvet}
\usepackage{listings}
\usepackage{tikz}
\usepackage{fontawesome}

\newcommand{\chref}[2]{%
  \href{#1}{{\usebeamercolor[bg]{AAUsimple}#2}}%
}

\newcommand{\FTdir}{}
\def\FTdir(#1,#2,#3){%
  \FTfile(#1,{{\color{black!40!white}\faFolderOpen}\hspace{0.2em}#3})
  (tmp.west)++(0.8em,-0.4em)node(#2){}
  (tmp.west)++(1.5em,0)
  ++(0,-1.3em) 
}

\newcommand{\FTfile}{}
\def\FTfile(#1,#2){%
  node(tmp){}
  (#1|-tmp)++(0.6em,0)
  node(tmp)[anchor=west,black]{\tt #2}
  (#1)|-(tmp.west)
  ++(0,-1.2em) 
}

\newcommand{\FTroot}{}
\def\FTroot{tmp.west}

\author{}
\institute{
  Institut for Matematiske Fag\\
  Aalborg Universitet\\
  Danmark

}
\pgfdeclareimage[height=1.5cm]{titlepagelogo}{AAUgraphics/aau_logo_new}
\titlegraphic{\pgfuseimage{titlepagelogo}}

\begin{document}
...
\end{document}
\end{verbatim}
  \end{block}
\end{frame}

\section{Fejlmeddelelser}

\begin{frame}[containsverbatim]{Fejlmeddelelser}{Typer}
  \begin{itemize}
  \item \LaTeX\ giver fejl hvis en kommando er skrevet forkert
  \item Det sker \textbf{hele tiden for os alle}
  \item Man skal derfor vide, hvordan man retter fejl
  \item En fejlmeddelelse starter altid med et udråbstegn
  \item Editors som TeXMaker samler fejlmeddelelser fra log-filen
  \item Brug Google - andre har helt sikkert haft den samme fejl
  \end{itemize}

\begin{verbatim}
! Too many }'s.
l.6 \frac 1}{2}

! Undefined control sequence.
l.6 \dtae

! Missing $ inserted
\end{verbatim}
\end{frame}

\begin{frame}[containsverbatim]{Fejlmeddelelser}{Strategi}
  \begin{itemize}
  \item Mange fejl kan fanges ved at kompilere ofte
  \item Nogle gange ved man ikke, hvor i dokumentet fejlen er
  \item Brug evt. udelukkelsesmetoden her---udkommenter indtil det virker igen, så ved du, at fejlen er i den udkommenterede del af koden
  \end{itemize}
\end{frame}

\section{Floats og Andre Blokmiljøer}

\begin{frame}{Floats og Andre Blokmiljøer}{Floats}

  \begin{block}{Floats}
    \begin{itemize}
  \item Floats bruges til indhold som ikke må brydes over flere sider
  \item De kan indeholde captions til at beskrive indholdet
  \item En caption \textbf{SKAL} kunne forklare indholdet uden man behøver læse brødteksten
  \item De er nummererede og man kan henvise til dem i brødteksten
  \end{itemize}
  \end{block}

  \begin{block}{Typiske Floats}
    \begin{itemize}
    \item Figurer
    \item Tabeller
    \end{itemize}
  \end{block}
\end{frame}

\begin{frame}[containsverbatim]{Floats og Andre Blokmiljøer}{Figurer}

\begin{columns}
  \begin{column}{0.6\textwidth}
    \begin{verbatim}
\begin{figure}[h] % options: h, t, b
 \centering
 \includegraphics{sti-til/taylor.pdf}
 \caption{Taylor.}
 \label{fig:taylor}
\end{figure}
\end{verbatim}
  \end{column}
  \begin{column}{0.4\textwidth}
\begin{figure}[h] % options: h, t, b
 \centering
 \includegraphics[scale=0.25]{img/taylor.pdf}
 \caption{Taylor approksimation af $f(x) = \log(x)$.}
 \label{fig:taylor}
\end{figure}
  \end{column}
\end{columns}



\end{frame}


\begin{frame}[containsverbatim]{Floats og Andre Blokmiljøer}{Tabeller}

\begin{columns}
  \begin{column}{0.6\textwidth}
    \begin{verbatim}
\begin{table}[h]
 \begin{center}
   \begin{tabular}{c|c|c}
    $n$ & $f^{(n)}(x)$ & $f^{(n)}(0)$\\
    \hline
    1 & $\log(x)$        & 0  \\[2pt]
    2 & $\frac{1}{x}$    & 1  \\[2pt]
    3 & $-\frac{1}{x^2}$ & -1 \\[2pt]
    \hline
   \end{tabular}
 \end{center}
 \caption{Taylor}
 \label{tab:taylor_lnx}
\end{table}
\end{verbatim}
  \end{column}
  \begin{column}{0.4\textwidth}
  \begin{table}[h]
  \begin{center}
    \begin{tabular}{c|c|c}
     $n$ & $f^{(n)}(x)$     & $f^{(n)}(0)$\\\hline
     1   & $\log(x)$        & 0           \\[2pt]
     2   & $\frac{1}{x}$    & 1           \\[2pt]
     3   & $-\frac{1}{x^2}$ & -1          \\[2pt]
      \hline
    \end{tabular}
  \end{center}
  \caption{Evaluering af $f$ og de afledte i $a=1$.}
  \label{tab:taylor_lnx}
\end{table}
\end{column}
\end{columns}
\end{frame}


\begin{frame}[containsverbatim]{Floats og Andre Blokmiljøer}{Andre Blokmiljøer}

\begin{columns}
  \begin{column}{0.6\textwidth}
  \begin{block}{Ikke-ordnede lister}
\begin{verbatim}
\begin{itemize}
 \item $\alpha$
 \item $\beta$
 \item $\zeta$
\end{itemize}
\end{verbatim}
  \end{block}
  \begin{block}{Ordnede lister}
\begin{verbatim}
\begin{enumerate}
 \item $\delta$
 \item $\epsilon$
 \item $\phi$
\end{enumerate}
\end{verbatim}
  \end{block}
  \end{column}
  \begin{column}{0.4\textwidth}
  \begin{block}{}
\begin{itemize}
 \item $\alpha$
 \item $\beta$
 \item $\zeta$
\end{itemize}
  \end{block}
  \begin{block}{}
\begin{enumerate}
 \item $\delta$
 \item $\epsilon$
 \item $\phi$
\end{enumerate}
  \end{block}
  \end{column}
\end{columns}
\end{frame}

\begin{frame}[containsverbatim]{Floats og Andre Blokmiljøer}{Krydsreference}
\begin{columns}
  \begin{column}{0.6\textwidth}
    \begin{verbatim}
\section{Krydsreferencer}
\label{krydsrefs}

Se table \ref{tab:taylor_lnx}
og figur \ref{fig:taylor}.

Betragt følgende
\begin{equation} \label{eq:1}
\sum_{k=0}^{n} k = \frac{n(n+1)^2}{2}
\end{equation}

Ligning \eqref{eq:1} blev opdaget
af Gauss i 1800 tallet.
\end{verbatim}
  \end{column}
  \begin{column}{0.4\textwidth}
    Se tabel \ref{tab:taylor_lnx}
    og figur \ref{fig:taylor}.

    \vspace{25pt}

    Betragt følgende
    \begin{equation} \label{eq:1}
      \sum_{k=0}^{n} k = \frac{n(n+1)^2}{2}
    \end{equation}

    Ligning \eqref{eq:1} blev opdaget
    af Gauss i 1800 tallet.
\end{column}
\end{columns}
\end{frame}


\section{Opgaver}

\begin{frame}[containsverbatim]{Opgaver}{Opgave 1}
  \begin{itemize}
  \item Lav en mappe struktur tilsvarende nedenstående
  \item Udfyld de fire \texttt{.tex} filer med relevant indhold
  \item Download filen \texttt{taylor\textunderscore graf.pdf}
  \item Inkluder \texttt{taylor\textunderscore graf.pdf} i \texttt{taylor\textunderscore tabel.tex}
  \end{itemize}
  \begin{tikzpicture}%
  \draw[color=black!60!white]
  \FTdir(\FTroot,root,opgave){       % root: parent = \FTroot
    \FTfile(root,master.tex)
    \FTfile(root,master.pdf)
    \FTdir(root,incl,incl){       % normal dir: (parentID, currentID, label)
      \FTdir(incl,fig,fig) {
        \FTfile(fig,taylor\textunderscore graf.pdf)
      }
      \FTdir(incl,kap,kap) {
        \FTfile(kap,taylor.tex)
      }
      \FTdir(incl,tab,tab) {
        \FTfile(tab,taylor\textunderscore tabel.tex)
      }
    }
  };
\end{tikzpicture}
\end{frame}

\end{document}

%%% Local Variables:
%%% mode: latex
%%% TeX-master: t
%%% End:
